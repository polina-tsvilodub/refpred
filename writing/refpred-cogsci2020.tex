% 
% Annual Cognitive Science Conference
% Sample LaTeX Paper -- Proceedings Format
% 

% Original : Ashwin Ram (ashwin@cc.gatech.edu)       04/01/1994
% Modified : Johanna Moore (jmoore@cs.pitt.edu)      03/17/1995
% Modified : David Noelle (noelle@ucsd.edu)          03/15/1996
% Modified : Pat Langley (langley@cs.stanford.edu)   01/26/1997
% Latex2e corrections by Ramin Charles Nakisa        01/28/1997 
% Modified : Tina Eliassi-Rad (eliassi@cs.wisc.edu)  01/31/1998
% Modified : Trisha Yannuzzi (trisha@ircs.upenn.edu) 12/28/1999 (in process)
% Modified : Mary Ellen Foster (M.E.Foster@ed.ac.uk) 12/11/2000
% Modified : Ken Forbus                              01/23/2004
% Modified : Eli M. Silk (esilk@pitt.edu)            05/24/2005
% Modified : Niels Taatgen (taatgen@cmu.edu)         10/24/2006
% Modified : David Noelle (dnoelle@ucmerced.edu)     11/19/2014
% Modified : Roger Levy (rplevy@mit.edu)     12/31/2018



%% Change "letterpaper" in the following line to "a4paper" if you must.

\documentclass[10pt,letterpaper]{article}

\usepackage{cogsci}

\cogscifinalcopy % Uncomment this line for the final submission 

\usepackage{xspace}
\usepackage{pslatex}
\usepackage{apacite}
\usepackage{float} % Roger Levy added this and changed figure/table
                   % placement to [H] for conformity to Word template,
                   % though floating tables and figures to top is
                   % still generally recommended!

%\usepackage[none]{hyphenat} % Sometimes it can be useful to turn off
%hyphenation for purposes such as spell checking of the resulting
%PDF.  Uncomment this block to turn off hyphenation.


%\setlength\titlebox{4.5cm}
% You can expand the titlebox if you need extra space
% to show all the authors. Please do not make the titlebox
% smaller than 4.5cm (the original size).
%%If you do, we reserve the right to require you to change it back in
%%the camera-ready version, which could interfere with the timely
%%appearance of your paper in the Proceedings.
\usepackage{multirow}
\usepackage{graphicx}
\usepackage{tabularx}
\usepackage{xspace}
%\usepackage{hyperref}
% packages to read R results 
\usepackage{pgfplotstable}
\usepackage{csvsimple}
\usepackage{siunitx}

%\newcolumntype{b}{X}
%\newcolumntype{s}{>{\hsize=2\hsize}X}


\definecolor{Red}{RGB}{255,0,0}
\definecolor{Green}{RGB}{10,200,100}
\definecolor{Blue}{RGB}{10,100,200}
\definecolor{Orange}{RGB}{255,153,0}
\definecolor{Purple}{RGB}{139,0,139}

\newcommand{\denote}[1]{\mbox{ $[\![ #1 ]\!]$}}
\newcommand*\diff{\mathop{}\!\mathrm{d}}
\newcommand{\red}[1]{\textcolor{Red}{#1}}  
\newcommand{\mht}[1]{\textcolor{Blue}{[mht: #1]}}  
\newcommand{\rl}[1]{\textcolor{Orange}{[rl: #1]}}  
\newcommand{\js}[1]{\textcolor{Green}{[js: #1]}} 
\newcommand{\pt}[1]{\textcolor{Purple}{[pt: #1]}} 

% define functions for reading results from csv
\newcommand{\datafoldername}{R_results_TeX}

% the following code defines the convenience functions
% as described in the main text below

% rlgetvalue returns whatever is the in cell of the CSV file
% be it string or number; it does not format anything
\newcommand{\rlgetvalue}[4]{\csvreader[filter strcmp={\mykey}{#3},
             late after line = {{,}\ }, late after last line = {{}}]
            {\datafoldername/#1}{#2=\mykey,#4=\myvalue}{\myvalue}}

% rlgetvariable is a shortcut for a specific CSV file (myvars.csv) in which
% individual variables that do not belong to a larger chunk can be stored
\newcommand{\rlgetvariable}[2]{\csvreader[]{\datafoldername/#1}{#2=\myvar}{\myvar}\xspace}

% rlnum format a decimal number
\newcommand{\rlnum}[2]{\num[output-decimal-marker={.},
                             exponent-product = \cdot,
                             round-mode=places,
                             round-precision=#2,
                             group-digits=false]{#1}}

\newcommand{\rlnumsci}[2]{\num[output-decimal-marker={.},
                          scientific-notation = true,
                             exponent-product = \cdot,
                             round-mode=places,
                             round-precision=#2,
                             group-digits=false]{#1}}

\newcommand{\rlgetnum}[5]{\csvreader[filter strcmp={\mykey}{#3},
             late after line = {{,}\ }, late after last line = {{}}]
            {\datafoldername/#1}{#2=\mykey,#4=\myvalue}{\rlnum{\myvalue}{#5}}}

\newcommand{\rlgetnumsci}[5]{\csvreader[filter strcmp={\mykey}{#3},
             late after line = {{,}\ }, late after last line = {{}}]
            {\datafoldername/#1}{#2=\mykey,#4=\myvalue}{\rlnumsci{\myvalue}{#5}}}

% MH's command
\newcommand{\brmresults}[2]{\(\beta = \rlgetnum{#1}{Rowname}{#2}{Estimate}{3}\) (\rlgetnum{#1}{Rowname}{#2}{l.95..CI}{3}, \rlgetnum{#1}{Rowname}{#2}{u.95..CI}{3})}
%\brmresults{expt1_brm.csv}{condition}


%\title{Inferring comparison classes from syntactic cues and world knowledge}
%\title{Informational goals constrain how listeners infer comparison classes}
\title{Informational goals, sentence structure, and comparison class inference}
% slightly misleading since we don't mention IS anymore.
%\author{{\large \bf Michael Henry Tessler (tessler@mit.edu)} \\
%  Department of Brain and Cognitive Sciences,  \\
%  Cambridge, MA  USA
%  \AND {\large \bf Sharon J.~Derry (SDJ@Macc.Wisc.Edu)} \\
%  Department of Educational Psychology, 1025 W. Johnson Street \\
%  Madison, WI 53706 USA}

 \author{{\large \bf Michael Henry Tessler}  \\  Brain and Cognitive Sciences \\   MIT  \\  \texttt{tessler@mit.edu}
         \And  
         {\large \bf Polina Tsvilodub}  \\ Institute of Cognitive Science \\   Osnabrück University  \\  \texttt{ptsvilodub@uos.de}
          \And
         {\large \bf Jesse Snedeker}  \\ Department of Psychology \\   Harvard University  \\  \texttt{snedeker@wjh.harvard.edu}
                   \And
         {\large \bf Roger Levy} \\ Brain and Cognitive Sciences \\   MIT  \\  \texttt{rplevy@mit.edu}}

\begin{document}

\maketitle

\begin{abstract}
%   Most Great Danes are big dogs, but some are also big Great Danes. Understanding a gradable adjective (e.g., \emph{big}) requires making reference to a \textit{comparison class}, a set of objects or entities against which the referent is implicitly compared (e.g., big for a Great Dane), but how do listeners decide upon a comparison class? Simple models of semantic composition stipulate that the adjective combines with a noun, which necessarily becomes the comparison class (e.g., ``That Great Dane is big'' means big for a Great Dane). We investigate an alternative hypothesis: the noun in a sentence is a cue to the comparison class, which must be integrated with other cues, like syntax, for a listener to infer the intended comparison class. We theorize that the utility of a noun in an adjectival utterance can be either for \textit{reference} (getting the listener to attend to the right object) or \textit{predication} (describing a property of the referent). Therefore, we hypothesize that when the presence of a noun can be explained by its utility in reference (e.g., being in the subject position: ``That NP is big''), it is less likely to set the comparison class; in contrast, nouns whose presence is more weakly explained by their utility in reference (e.g., predicate-NPs: ``That’s a big NP'') are more likely to set the comparison class. Across three pre-registered experiments, we find evidence that listeners integrate the noun of a sentence with syntactic information and world knowledge to infer comparison classes, consistent with a trade-off between reference and predication.
   
Understanding a gradable adjective (e.g., \emph{big}) requires making reference to a comparison class, a set of objects or entities against which the referent is implicitly compared (e.g., big for a Great Dane), but how do listeners decide upon a comparison class? Simple models of semantic composition stipulate that the adjective combines with a noun, which necessarily becomes the comparison class (e.g., ``That Great Dane is big'' means big for a Great Dane). We investigate an alternative hypothesis built on the idea that the utility of a noun in an adjectival utterance can be either for reference (getting the listener to attend to the right object) or predication (describing a property of the referent). Therefore, we hypothesize that when the presence of a noun (N) can be explained away by its utility in reference (e.g., being in the subject position: ``That N is big''), it is less likely to set the comparison class. Across three pre-registered experiments, we find evidence that listeners integrate the noun of a sentence with syntactic information and world knowledge to infer comparison classes, consistent with a trade-off between reference and predication. 
 
   
   
\end{abstract}

\begin{quote}
\small
\textbf{Keywords:} 
comparison class; adjectives; information structure; reference; predication
\end{quote}
%\section{Introduction}
The meanings of linguistic expressions can change dramatically depending on the context. 
But determining which aspects of context are relevant for understanding a speaker's message is far from understood.
This issue is brought into focus when trying to understand gradable adjectives like \emph{big}, \emph{tall}, or \emph{beautiful}.
The utterance ``That Great Dane is big'' informs the listener that the referent (a Great Dane) has a relatively large size, but relative to what the speaker thinks the Great Dane is big goes unsaid: The Great Dane could be \textit{big for a Great Dane},  \textit{big for a dog}, \textit{big for a four-legged creature}, as well as an infinity of other possibilities. 
How do human listeners determine the comparison class when faced with multiple \textit{a priori} reasonable options? 

Simple models of semantic composition posit that when an adjective combines syntactically with a noun (N), an interpretable adjectival phrase is produced by the noun providing the comparison class \cite<e.g., big(car) $\rightarrow$ \textit{big for a car},  small(watch) $\rightarrow$ \textit{small for a watch};>{Kamp1975, Cresswell1976}. 
Many arguments have been laid against such a simple mapping between the noun in the sentence and the comparison class \cite<e.g.,>{Bierwisch1989, Kennedy2007}: A \emph{big snowman} said of a snowman that a 4-year-old built probably means something like \textit{big relative to snowmen that 4-year-olds can build}; a \emph{rich Fortune-500 CEO} might not be \textit{rich relative to other Fortune-500 CEOs}. 
Theoretical work on comparison classes has focused on how comparison classes are integrated into a compositional semantics and what representations might be preferred \cite{Kennedy2007, Solt2009, Bale2011}.
Yet, little is known about how human listeners decide upon a comparison class in context. 
\begin{figure*}[t]
\begin{center}
\includegraphics[width=0.9\linewidth]{ref-pred-cartoon-w-subscripts2.pdf}
\end{center}
\caption{Cartoon of inferential account for comparison class determination. The noun (Great Dane) in a sentence can be employed either for the goal of reference (green) or predication (purple), shown in the case when this distinction is made via the syntactic position of the N (subject (S)~vs.~predicate (P)). When the noun is used for reference (top), a listener is left with uncertainty about what to use as the comparison class (dogs or Great Danes) and integrates their world knowledge and the physical context to make this inference.  When the noun is used for predication (bottom), the listener should have less uncertainty about the comparison class: The comparison class is stipulated by the noun.}
\label{model-cartoon}
\end{figure*}

We examine the problem from a functional perspective: what goals are speakers trying to achieve when crafting their utterance, and how might these goals influence listeners’ interpretations?  % should we include these definitions here?? (they occur in the abstract before)
In order to communicate a property of a referent, a speaker must achieve two informational goals: the goal of \emph{reference} (helping the listener identify the right target) and the goal of \emph{predication} (communicating a property of the referent) \cite{Reboul2001}. 
In simple, subject-predicate sentences of the form ``$S$ $P$'', where $S$ is a referential subject noun phrase and $P$ is a predicate that is asserted to hold of the subject, we posit that listeners expect that the referent will be made clear by the subject N -- independent of the predicate -- and that speakers aim to satisfy this expectation.
\footnote{Of course, it is not universally true that the referent is established by the subject NP (e.g., insofar as one can infer who \emph{he} is in the sentence ``He's making those outrageous tweets again.'', it is because the predicate provides a cue to the referent). We posit this relation between subject NP and reference as an expectation that listeners may hold, perhaps due to information structural reasons.
}
\pt{subject NOUN PHRASE might be unclear, since we also look at sentences where the subject has no N and posit the same; we might want to add a sentence about general expectations of reference-in-subject. maybe sth like this? For simple sentences of the form "Subject Predicate", we posit that listeners expect that the referent will be made clear by the Subject - independently of the predicate, which is asserted to hold of the subject, and that speakers... }  
%We explore the implications of such a theory of 
%In this paper, we
%
%
% (e.g., the Great Dane over there by that park bench is big)
%
%We propose an inferential theory of comparison class determination: Speakers aim to achieve basic informational goals that guide how they structure their utterance, and listeners infer the most likely comparison class in light of a speaker's goals.  
%In order to successfully predicate something, one must establish reference.
%In order to successfully convey a speaker’s intended message, the $S$ $P$ sentence must clearly establish the referent of the subject NP \mht{$\leftarrow$  does this first clause already presuppose our posit that comes after (that the referent of the subject NP will be clear independently of the predicate)? i get a little confused by the phrase ``referent of the subject NP''. i feel like there is a referent, and there is a subject NP, and the subject NP is used to establish the referent in context...};
%In order to predicate something, you need to predicate of something.
%Listeners expect this to happen (the rference) in the subject... (this is an inforamtion structure point)
If this expectation holds when interpreting gradable adjectives, then the influence of a speaker’s choice of noun on the comparison class will depend on whether the noun appears in the subject (for the goal of reference) or the predicate (for the goal of predication; Fig.~\ref{model-cartoon}).
If the noun appears in the predicate (``That’s a big \{Great Dane, dog\}''), then it is natural to explain the speaker’s choice of noun as non-referential, but rather as a cue to the intended comparison class.  
In contrast, if the noun appears in the subject (``That \{Great Dane, dog\} is big''), then the speaker’s choice of noun can potentially be \emph{explained away} as intending to help the listener establish reference of the subject, thus serving as a weaker cue to comparison class and allowing other pragmatic reasoning (e.g., world knowledge and perceptual cues) to play a larger role in determining the comparison class \cite<e.g., the Great Dane is big for a dog;>{tessler2017warm}.
%We examine the problem from a functional perspective---what goals are speakers trying to achieve when crafting their utterance?---and focus on the role of the noun phrase (NP) in determining the comparison class (Figure \ref{model-cartoon}). %and derive novel predictions about comparison class inferences via the interaction of syntactic cues, world knowledge, and the perceptual context.
%Functionally, noun phrases (e.g., \emph{Great Dane}) can be used both for achieving the goals of \emph{reference} (i.e., getting the listener to attend the object that the speaker intends; e.g., ``Look at that Great Dane.'') and \emph{predication} (i.e., describing a property of that referent; e.g., ``This is a Great Dane.''; \cite{Reboul2001}.\footnote{The reference-predication distinction is similar to the topic--comment or given--new distinction in Information Structure \cite{lambrecht1996information}. We return to this point in the discussion.}
%When an NP appears in a sentence with a scalar adjective, the goal of predication can be understood as indicating the comparison class by communicating the speaker's perspective on the referent; a speaker using an NP to satisfy the goal of predication (e.g., via the NP appearing in the predicate of the sentence as in ``That's a big Great Dane'') can thus serve as a cue that the NP is the comparison class (i.e., \emph{it's big relative to other Great Danes}). 
%On the other hand, the presence of an NP in a sentence could also be explained by the goal of reference (e.g., by appearing in the subject of the sentence as in ``That Great Dane is big''), in which case the NP would less strongly constrain the comparison class and other pragmatic reasoning (e.g., with world knowledge) would serve to fill in the comparison class \cite<e.g., the Great Dane is big for a dog;>{tessler2017warm}.
We test this reference -- predication trade-off hypothesis using a syntactic manipulation wherein the N can appear either in the subject or the predicate of a sentence involving a gradable adjective (e.g., ``That Great Dane is big''~vs.~``That's a big Great Dane''). 

We note that this subject-predicate distinction could also be viewed from a syntactic-modification perspective, wherein the comparison class would be determined by the syntactically-modified N, as opposed to the unmodified N which wouldn't affect the comparison class. However, such an account would stipulate that modified nouns (i.e., predicate-Ns) always map onto a comparison class, as well as that inferred comparison classes wouldn't be affected by distinct subject-Ns. Finally, if the information needed to determine the comparison class is all found in the words of the sentence, then the same sentence in a different context should not change the comparison class. 

The critical test of this syntactic manipulation is how speakers and listeners treat these sentences in the context of a referent for whom the adjective is felicitous given one comparison class but not another (e.g., \emph{big} to describe a normal-sized Great Dane, which would be big if the comparison class is \emph{dogs} but not \emph{Great Danes}). 
We examine human judgments using three distinct dependent measures in pre-registered experiments. 

 %Across these diverse measures, we find consistent evidence for a reference-predication trade-off guiding comparison class inferences when interpreting adjectival sentences.
% Maybe use this and drop the expt. description part in intro of Experiments?

%In Experiment 1 (Syntax Rating), participants provide acceptability ratings of the syntactically-distinct sentences. In Experiment 2 (NP Production), participants produce noun phrases in different syntactic frames (e.g., ``That's a big \_\_'' vs. ``That \_\_ is big''). Finally, in Experiment 3 (Comparison Class Inference), participants paraphrase a speaker's adjectival sentence with an explicit comparison class (e.g., ``It's big relative to other \_\_''), building on the empirical paradigm of \citeA{tessler2017warm}.

%(e.g., \emph{I see this thing as a dog}) and thus should serve as a strong cue to the comparison class in case of gradable adjectives (i.e., the Great Dane is \emph{big for a Great Dane})

%to predict interpretative differences between \emph{attributive} uses of adjectives (e.g., ``That's a big Great Dane'') and \emph{predicative} uses (e.g., ``That Great Dane is big'') . 
%For attributive uses, the NP is likely to be used for predication, i.e. communicating the speaker's perspective
% conceptualization of
%on the referent (e.g., \emph{I see this thing as a Great Dane}) and thus should serve as a strong cue to the comparison class in case of gradable adjectives (i.e., the Great Dane is \emph{big for a Great Dane}). % somewhat unclear to me, see below
%In predicative uses, on the other hand, the NP is likely to be used for reference (e.g., an NP combined with a deictic such as ``That Great Dane is big''), and thus, should less strongly constrain the comparison class.
% maybe we could start with predicative adjectives because in general reference needs to be established first? And we were arguing about the attributive cases as being non-explainable in terms of reference and hence providing the CC?
%When the NP provides a weak cue to the comparison class, world knowledge can more strongly influence the comparison class \cite<e.g., the Great Dane is big for a dog;>{tessler2017warm}.
%In line with definitions of a predication act as proposed in the literature \cite{Reboul2001}, we assume that listeners always first need to establish reference before interpreting the predicate. In predicative adjective uses, the NP is considered to more likely be referential since it is combined with the deictic 'That' and constitutes a definite description. However, Donnellan proposes a distinction of definite descriptions into attributive and referential cases, where the attributive definite descriptions (non-referring) are those where the identity of the referent is unknown \cite{Reboul2001}, which supports our assumptions that the referential utility of the NP may vary. The referential vs. attributive use may be resolved pragmatically. In attributive constructions we consider, the NP is more likely to be non-referential since it is an indefinite description. NPs which do not contribute to reference are thought to "contribute bound variables and predicates" to proposition \cite{Reboul2001}. 
%We build upon the Rational Speech Act framework -  a recursive Bayesian model where the speaker and the listener coordinate an intended meaning - and suggest that listeners integrate this implicit grammatical knowledge with their world knowledge about categories and their properties and perceptual cues to infer the correct comparison class. 
% that two fundamental, informational goals of communication can serve as a basis for inferring a speaker's intended comparison class. 

\section{Experiments}
Our guiding hypothesis is that when speakers compose their utterance, the utility of a N in reference trades off with the utility of it conveying a feature value of the referent (predication) \footnote{For scalar degrees, this amounts to communicating the intended comparison class of the respective gradable adjective}; utility in reference can then ``explain away'' the utility of using a noun to set the comparison class. 
We operationalize utility in reference via the syntactic frame in which the noun phrase appears: if the noun appears in the subject of the sentence (That N is ADJ), it is likely to be used for reference and less likely to set the comparison class. If the noun appears in the predicate of the sentence (That's an ADJ N), it is unlikely to be used for reference and more likely to set the comparison class. 

In all of our experiments, we use the ADJs \emph{big} and \emph{small} because of the simplicity with which the feature value (i.e., size) can be conveyed through visual presentation and for which people have strong expectations about the feature value for members of different categories (e.g., Great Danes are generally big dogs; goldfish are generally small fish; etc.). 
Referents were always described using the size adjective consistent with these general expectations (e.g., \emph{Great Dane -- big}, \emph{goldfish -- small}), to allow for the possibility of either subordinate (Great Dane) or basic-level (dog) comparison classes.
%Thus, given world knowledge, both the subordinate-level or basic-level comparison class could be felicitous.%, though the basic-level is more likely \textit{a priori} (i.e., the Great Dane is likely to be \textit{big for a dog}, but could also be \textit{big for a Great Dane}).
The preregistrations and full experimental procedures can be viewed at \url{tinyurl.com/rcsyz9f}\footnote{All data and code can be found under \url{https://github.com/polina-tsvilodub/refpred}}.

%; thus, referential utility can serve as a cue to the comparison class for listeners and a guide for speaker preferences about  . %
%  which we examine using the subject vs. predicate contrast in where the NP appears. 


%In Experiment 1 (Syntax Rating), participants provide acceptability ratings of the syntactically-distinct adjectival sentences to describe an object for which the size adjective would be felicitous given one comparison class but not another. In Experiment 2 (NP Production), tests if participants produce different nouns given different syntactic frames (e.g., ``That's a big \_\_'' vs. ``That \_\_ is big''). Finally, Experiment 3 (Comparison Class Inference) investigates listeners' comparison class inferences as driven by the syntax, the noun phrase, and the immediate perceptual context in which the sentence is uttered, via participants paraphrasing a speaker's adjectival sentence with an explicit comparison class (e.g., ``It's big relative to other \_\_''), building on the empirical paradigm of \citeA{tessler2017warm}.
%Experiment 1 tests if participants prefer one noun position in the sentence over the other to describe an object for which the size adjective \emph{big} would be felicitous given one comparison class but not another. Experiment 2 tests if speakers produce different nouns in different syntactic frames to describe the same object. Finally, Experiment 3 investigates listeners' comparison class inferences as driven by the syntax, the noun phrase, and the immediate perceptual context in which the sentence is uttered.


\begin{figure*}[t]
\begin{center}
\includegraphics[width=\textwidth]{screenshots.pdf}
\end{center}
\caption{Overview of Experiments 1-3. A - B: Example context stimuli. A: Basic-level contexts used in Expts.~1-3. B: Subordinate context from Expt.~3. C - E: Example test questions with referents. C: Syntax Rating trial (Expt.~1) with a referent from a large-subordinate category referred to with a subordinate N. D: NP Production trial (Expt.~2) with a referent from a small-subordinate category described with a predicate-N syntactic frame. E: Comparison Class Inference trial (Expt.~3) with a referent from a large-subordinate category described with a subject-N syntactic frame using a subordinate-N label.} 
\label{screenshots}
\end{figure*}
\begin{table}[t]
\small{
\begin{center}
\caption{Experimental items: each basic-level context had two potential targets from an either saliently small or saliently big subordinate category within the basic-level class. Items marked with * were used in Expt. 2.}
\label{tab:stimuli}
\vskip 0.12in
\fontsize{10}{11}\selectfont
\begin{tabularx}{\linewidth}{lll}
\hline
 Basic-level category & Smaller referent & Bigger referent\\
\hline
 Dogs & Pug & Great Dane \\
 Dogs & Chihuahua & Doberman\\
 Birds & Hummingbird & Eagle  \\
 Fish & Goldfish & Swordfish \\
 Flowers & Dandelion & Sunflower\\
 Trees & Bonsai & Redwood\\
Birds* & Sparrow* & Goose* \\
Birds* & Canary* & Swan* \\
Fish* & Clownfish* & Tuna* \\
Flowers* & Daisy* & Peony* \\
\hline     
\end{tabularx}
\end{center}
}
\end{table}

\subsection{Experiment 1: Syntax Rating} In this experiment participants rated how well each of two sentences differing in the position of the noun described the target in context. The noun was either the basic-level (e.g., dog) or the subordinate target label (e.g., Great Dane; within-subjects).



\subsubsection{Participants} We recruited \rlgetvariable{myvars-rating.csv}{nSubj} participants from Amazon's Mechanical Turk; participants in all experiments were  restricted to those with US IP addresses and at least a 95\% work approval rating. We excluded \rlgetvariable{myvars-rating.csv}{nExcludedTotal} for self-reporting a native language other than English, for failing a comprehension check or providing the same responses on every trial.  The experiment took about 5 minutes and participants were compensated \$0.80. %\rlgetvariable{myvars-rating.csv}{nFailedWarmUp}  and \rlgetvariable{myvars-rating.csv}{nFailedMains} for  


\subsubsection{Materials} 
All experiments used the same materials. 
%We used the positive- and negative-form gradable adjectives describing size: \emph{big} and \emph{small}. %In all experiments, the critical sentence(s) with the gradable adjective describe a target object presented visually alongside other objects (the \emph{visual context}; Figure~\ref{screenshots}A,B).
Nouns and referent pictures were chosen from five \emph{basic-level categories} in the animal and plant domains: dogs, birds, fish, flowers, trees.
Within each basic-level category, we chose target objects from \emph{subordinate level categories} about which people have prior expectations concerning the size of members of those categories (Table \ref{tab:stimuli}).
%For example, Great Danes are generally big relative to other dogs; goldfish are generally small relative to other fish. 
%Targets are described using the size adjectives consistent with these general expectations (e.g., \emph{Great Dane -- big}, \emph{goldfish -- small}).
%Thus, given world knowledge, both the subordinate-level or basic-level comparison class could be felicitous.%, though the basic-level is more likely \textit{a priori} (i.e., the Great Dane is likely to be \textit{big for a dog}, but could also be \textit{big for a Great Dane}).

\subsubsection{Procedure}
Participants completed two comprehension check trials and six main trials. In the comprehension check trials, participants see a picture (e.g., a purple chair), read pairs of sentences describing it (e.g., ``The chair is blue'' and ``The chair is yellow''), and are asked to rate on slider how well each of the sentences describes the referent.

In the main trials, participants read: ``You and your friend see the following:'' above a context picture with other members of the same basic-level category (e.g., a group of dogs; Figure~\ref{screenshots}A). 
Six different basic-level contexts were created from the five categories depicting groups of several members belonging to different subordinate categories (e.g., dogs of different breeds, including the target and filler subordinate categories, such as Great Danes, pugs and poodles; Table~\ref{tab:stimuli}).
Below the context they read ``You also see this \emph{subordinate label}'' and saw the referent pictured.
% possibly cut the following paragraph?
%The visual size of the picture of the referent is manipulated such that it appears incongruent with the general expectations of that subordinate category (i.e., a Great Dane, which we would expect to be large, is actually small relative to other Great Danes in the context; Fig~\ref{screenshots}C compared to context in Fig.~\ref{screenshots}A). 
%This subtle visual disparity was imposed in order to enhance the difference in felicity between the adjective used with a basic-level~vs.~subordinate level comparison class (e.g., the Great Dane is \emph{big for a dog} but probably not \emph{big for a Great Dane}).

Participants rated how well two sentences described the target, using sliders ranging from \textit{very bad} to \textit{very well}. The sentences differed in whether the N appeared in the subject or predicate of the sentence (e.g., Predicate NP: ``That's a big Great Dane''; Subject NP: ``That Great Dane is big''; Fig.~\ref{screenshots}C), and the order in which the sentences and corresponding sliders appeared on the page was randomized between-subjects. 
Trials differed in whether the noun was the subordinate referent label (e.g., \emph{Great Dane}) or the basic-level label (e.g., \emph{dog}), in randomized order. 
Each participant saw only one of the two possible targets for each context (e.g., either the Great Dane or the pug for the dog basic-level context).

\begin{figure}[t]
\begin{center}
\includegraphics[width=\linewidth]{expt-syntax-rating-prereg-bars-revised.pdf}
\end{center}
\vspace{-1cm}
\caption{Experiment 1: Means and 95\% bootstrapped confidence intervals (bootstrapping independent of random-effects structure) of ratings of how well the sentences described the referent when different nouns (color) appeared in different syntactic frames (x-axis).  Points represent participant means within condition.
}
%\vspace{-0.2cm}
\label{syntax-rating}
\end{figure}

\subsubsection{Results} 
We found no effect of the slider presentation order (syntactic conditions), so the data was collapsed across the two conditions for all analyses. 
%shows the mean ratings provided for subordinate and basic-level NPs in the subject and predicate-NP syntactic frames.
Consistent with our prediction,  participants substantially dispreferred sentences with the subordinate noun in predicate position compared to the subject position (Figure~\ref{syntax-rating}), confirmed by a Bayesian generalized linear mixed-effects model with main effects of syntax, the noun phrase, and their interaction, as well as a maximal random effects structure.\footnote{In lmer-style syntax: \texttt{rating $\sim$ syntax * NP + (1 + syntax*NP | subject) + (1 + syntax*NP | target)}}
We found an interaction between the syntax and the NP  (mean and 95\% Bayesian credible interval: $\beta = \rlgetnum{expt1_brm.csv}{Rowname}{syntax:NP}{Estimate}{2}  [\rlgetnum{expt1_brm.csv}{Rowname}{syntax:NP}{l.95..CI}{2}, \rlgetnum{expt1_brm.csv}{Rowname}{syntax:NP}{u.95..CI}{2}]$), as well as an overall preference for the basic-level NPs ($\beta = \rlgetnum{expt1_brm.csv}{Rowname}{NP}{Estimate}{2} [\rlgetnum{expt1_brm.csv}{Rowname}{NP}{l.95..CI}{2},\rlgetnum{expt1_brm.csv}{Rowname}{NP}{u.95..CI}{2}] $) and subject-NP syntax ($\beta = \rlgetnum{expt1_brm.csv}{Rowname}{syntax}{Estimate}{2} [\rlgetnum{expt1_brm.csv}{Rowname}{syntax}{l.95..CI}{2}, \rlgetnum{expt1_brm.csv}{Rowname}{syntax}{u.95..CI}{2}] $). 
In exploratory analyses, we
% found no effects of the target size (Great Dane ~vs.~ pug). We 
observed considerable variation in the by-target intercepts (e.g., \emph{sunflower} item receives overall lower ratings), probably due to a varying basic-level label bias of the single items (the subordinate labels were more salient for some items than for others; $\beta = \rlgetnum{expt1_random_brm.csv}{Rowname}{sd(Intercept)}{item.Estimate}{2} [\rlgetnum{expt1_random_brm.csv}{Rowname}{sd(Intercept)}{item.l.95..CI}{2}, \rlgetnum{expt1_random_brm.csv}{Rowname}{sd(Intercept)}{item.u.95..CI}{2}]$). 
% I did an lmer analysis with size effect, the main effect was p=0.047, the syntax-NP-size interaction p=0.021
%Participants discriminate the felicity of sentences involving the same NPs and same adjectives, depending on the syntactic position of the NP. 
%This result is consistent with the hypothesis that the syntactic position of the NP modulates the strength of the cue that the NP provides towards the comparison class. 

\subsection{Experiment 2: Free-production of noun}
If the syntactic position of the N modulates the N-cue strength towards the comparison class, we would also expect speakers to produce different nouns depending on the syntactic position of the N, which we tested here.

\subsubsection{Participants}
We recruited \rlgetvariable{myvars-np.csv}{nSubj} participants and excluded \rlgetvariable{myvars-np.csv}{nExcludedTotal} for implementation glitches, native languages other than English or failing warm-up trials more than 4 times after correction. The experiment took about 7 minutes and participants were compensated \$1.00. %\rlgetvariable{myvars-np.csv}{nNonEN} for  and \rlgetvariable{myvars-np.csv}{nFailedWarmUp} for  

\subsubsection{Procedure}

The main trials were divided into two blocks, and before each block, participants completed warm-up trials.
The warm-up trials were designed to elicit category labels at different levels of abstraction (e.g., ``Great Dane'', ``pug'', ``dog'') by filling-in labeling sentences, for which they were provided corrective feedback.
%Two sentences appeared below each target reading ``This is a \_\_'', and one sentence appeared at the bottom of the page reading ``These are both \_\_.''
%Participants supplied an incorrect label, they were provided the correct label and were required to correct their response before proceeding.
The same subordinate referents were used as targets in the main trials. 
Trial order within each warm-up and main block was randomized. 
We used the same contexts as in Experiment 1 and created four additional basic-level contexts (Table \ref{tab:stimuli}). 
Six contexts were randomly sampled for each participant (three per block).

On the main trials, subjects saw “You see the following:” above the context picture (as in Expt.~1; Fig~\ref{screenshots}A). Below, they read “You also see this one:” and saw the picture of the referent (e.g., a Great Dane or a pug). They were told “You say to your friend:”, followed by either a subject-N or predicate-N sentence frame (between-subjects), where the noun was omitted (e.g., “That \_\_ is big“~vs.~“That’s a big \_\_ ''; Fig.~\ref{screenshots}D).
Each participant saw only either the big or the small target for each basic-level category.
The free-production responses were categorized by hand into subordinate or basic-level labels of the referent. 
16 uncategorizable responses (1.4\%) were excluded from the analysis.

\begin{figure}[t]
\begin{center}
\includegraphics[width=0.7\linewidth]{expt-np-prod-prereg-bars-revised.pdf}
\end{center}
\vspace{-1cm}
\caption{Experiment 2: Means and 95\% bootstrapped confidence intervals of produced basic-level labels (e.g., \emph{dog} when the referent was a Great Dane) in different syntactic frames (x-axis).}
\label{np-production}
\vspace{-0.2cm}
\end{figure}
\subsubsection{Results}
Participants produced basic-level nouns at a higher rate in the predicate than in the subject position (Figure \ref{np-production}), confirmed by a logistic Bayesian mixed-effects regression model, predicting the response category (basic-level vs. subordinate) by an intercept, the main effect of syntax %(contrast coded, subject vs. predicate NP) 
and by-participant and by-referent random intercepts and a by-referent random slope effect of syntax.\footnote{In lmer syntax: \texttt{response\_category $\sim$ syntax + (1 | subject) + (1 + syntax | target)}} Participants were appreciably more likely to use basic-level labels in the predicate position ($\beta = \rlgetnum{expt2_brm.csv}{Rowname}{syntax_contr}{Estimate}{2} [\rlgetnum{expt2_brm.csv}{Rowname}{syntax_contr}{l.95..CI}{2}, \rlgetnum{expt2_brm.csv}{Rowname}{syntax_contr}{u.95..CI}{2}]$). 
%In exploratory analyses, we did not find effects of target size (e.g., Great Dane vs. pug). 
%Similar to Expt.~1, exploratory analyses show considerable by-target variation in terms of the random intercept ($\beta = \rlgetnum{expt2_random_brm2.csv}{Rowname}{Intercept}{target.Estimate}{2} [\rlgetnum{expt2_random_brm2.csv}{Rowname}{Intercept}{target.l.95..CI}{2}, \rlgetnum{expt2_random_brm2.csv}{Rowname}{Intercept}{target.u.95..CI}{2}] $), likely due to differing subordinate-label accessibility (e.g., \emph{swan} is more accessible than \emph{peony}). Within the predicate noun trials, small targets elicited significantly more basic-level labels than big targets (e.g., pug $\rightarrow$ \emph{small dog} more so than Great Dane $\rightarrow$ \emph{big dog}; $\beta = \rlgetnum{expt2_predicate_brm.csv}{Rowname}{size_contr}{Estimate}{2} [\rlgetnum{expt2_predicate_brm.csv}{Rowname}{size_contr}{l.95..CI}{2}, \rlgetnum{expt2_predicate_brm.csv}{Rowname}{size_contr}{u.95..CI}{2}]$). % when subsetting the data by syntax, the re is a size effect in the predicate condition (beta = 1.61 [0.26, 3.21]).    
%Participants are more likely to choose the NP corresponding to the felicitous comparison class in the predicate syntactic frame. %produce an NP corresponding to the more felicitous -- basic-level -- comparison class in the predicate syntactic frame.
\subsection{Experiment 3: Comparison Class Inference}% In this experiment we elicited comparison classes inferred by listeners given different visual contexts, noun, and syntactic frames of the critical utterance (within-subjects). 
According to our inferential account, comparison class inferences should be driven by the noun (\emph{dog} or \emph{Great Dane}) to the extent that the usage of the noun cannot be explained away as achieving the goal of reference.
Our first two experiments support this view: Participants dispreferred sentences like ``That's a big Great Dane'' when the referent was not big for a Great Dane.
In this experiment, we provide a more direct test of our account by explicitly measuring comparison class inferences.

Our experimental design manipulates three factors within-subjects: syntactic position of the noun, the level-of-abstractness of the noun (basic-level vs. subordinate-level vs. neutral; e.g., \emph{dog} vs. \emph{Great Dane} vs. \emph{one}), and the visual context (e.g., other dogs vs. other Great Danes).
The inferential account provides a natural avenue for visual context to influence comparison class inferences.
When the noun usage can be explained away by its utility in reference, the account predicts that comparison class inferences should be driven by other factors (e.g., world knowledge or the visual context). 
By contrast, a purely syntactic-based account would predict no effect of visual context. %: If the information needed to determine the comparison class is all found in the words of the sentence, then the same sentence in a different context should not change the comparison class. 
Finally, we include a neutral noun condition using the anaphoric ``one'' to provide a base-line measure of the influence of visual context on comparison class inferences: In a context with varying types (basic-level context; Fig.~\ref{screenshots}A), anaphoric ``one'' should be interpreted as ``dog''; when the context provides animals of the same type (subordinate-level context; Fig.~\ref{screenshots}B), anaphoric ``one'' is more likely to be interpreted more narrowly (``Great Dane''). \pt{cite Goldberg?} 

%When the noun does contribute to the goal of reference, we predict comparison class inferences should be driven by the visual context.

\subsubsection{Participants} We recruited \rlgetvariable{myvars-infer.csv}{nSubj} participants and excluded \rlgetvariable{myvars-infer.csv}{nExcludedTotal} for either reporting other native languages than English, failing a task comprehension check, or failing warm-up trials more than 4 times after feedback. The experiment took about 9 minutes and participants were compensated \$1.20. % \rlgetvariable{myvars-infer.csv}{nFailedCCWarmUp} for d \rlgetvariable{myvars-infer.csv}{nFailedWarmUp} for  

\subsubsection{Procedure} Before the main trials, participants completed a comparison class paraphrase of the kind used in the main trials, for which they were provided corrective feedback.
%: they rephrased what they understand a speaker to mean by using an explicit comparison class. Participants read that a speaker said ``The Empire State Building is tall'' and were asked what the speaker meant: ``The Empire State Building is tall relative to other \_\_''. 
%Participants had to correct their response if they provided an infelicitous comparison class. %(viable responses included: buildings, skyscrapers, constructions, houses).
Following this comprehension test, participants completed two blocks of warm-up and main trials, akin to  Expt.~2.

In a main trial, participants read “You and your friend see the following:” above an either subordinate-level or basic-level context picture (Fig.~\ref{screenshots}A, B). 
Below the context picture, they read “Your friend runs far ahead of you, and you see him in the distance” and saw a cartoon of a person standing next to the referent (e.g., a Great Dane) in the distance so that the referent size could not be judged visually (Fig.~\ref{screenshots}E). 
Participants read ``Your friend says: [\emph{critical sentence}],'' which could vary by both syntactic position of the N (subject-~vs.~predicate-N) as well as the noun label. 
The noun label could be the subordinate target label (e.g., Great Dane), basic level label (dog), or the underspecified anaphoric \emph{one} (e.g., ``That one is big''). 
%We used \emph{one} in order to measure the baseline effect of visual context on comparison class inferences. %and to test whether listeners infer the comparison class from visual context and world knowledge (i.e. that the basic-level comparison class is a priori more likely to be felicitous) if the NP does not provide informative cues towards the comparison class. Therefore, we expected the comparison class to be the basic-level category of the referent given basic-level context (e.g. dogs) and the subordinate category given the subordinate context (e. g. great danes), though we expected to possibly see a basic-level bias, such that “dogs” was the a priori preferred comparison class..
Participants were asked “What do you think your friend meant?”, to which they responded in the sentence frame: “It is \{big, small\} relative to other \_\_” with whatever they thought made the most sense given the context (Fig.~\ref{screenshots}E).
Participants completed 12 trials, seeing exactly one trial in each condition (syntactic frame [subject vs. predicate], visual context [subordinate vs. basic], noun [subordinate vs. basic vs. \emph{one}]), the order of which was randomized.\pt{this is not completely correct since i messed up the randomization..}.

Our design attempts to isolate the sources of information that could be used for the comparison class inference. 
In particular, we wanted to remove the actual size of the referent as a potential source of information. 
Thus, we intentionally presented the friend/speaker as ``far ahead of you'' so that the size of the friend would not be a cue to the size of the referent, which could be used to determine the comparison class. 

%with the constraint that both small and big referents of a basic-level category (e.g., Great Danes and pugs) appeared in the same block, with one appearing in the basic-level visual context and the other appearing in its subordinate-level visual context (e.g., Great Dane appears with other dogs and the pug appears with other pugs). 

%\subsubsection{Predictions} %We predict comparison class inferences as a result of multiple cues operating in distinct manners. %This should be modulated by the syntactic position of the NP, with %In the abstract, comparison class inferences should be driven a basic-level bias \cite{rosch1975} %We predict referential utility of an NP should influence comparison class inferences. %In the basic-level visual context, we expected the subordinate NP to signal a subordinate comparison class more strongly when it appeared in predicate position than in the subject; we use the underspecified-NP (``one'') as the baseline comparison.  %In a subordinate visual context, the subordinate NP did not restrict the comparison class more beyond the context if the visual context pushed the basic-level category to floor, so the inferred comparison classes should correspond to baseline; if, however, the baseline comparison class inferences for the basic-level category were not at floor with the subordinate visual context, we expected the predicate syntax with the subordinate NP to further restrict the comparison class to the subordinate category. Similarly, the basic-level NP in the basic-level context did not restrict the comparison class more than the context, such that should be no differences in the basic-level inference proportions compared to the baseline (inferences drawn from ‘one’). In the subordinate context, however, the basic-level NP provided a more general comparison class than the context, such that the basic-level inference proportion should be higher given the predicate NP than the baseline; there might be no difference between the predicate and the subject basic-level NP since the subject NP cannot be explained by the goal of reference - its referential utility is low (all the distractors are dogs) and hence the NP was explained by predication - communicating the comparison class. %In the main trials, participants saw both possible subordinate targets within a basic-level category (e.g. both a great dane and a pug for dogs) (Table \ref{tab:stimuli}). \mht{<-- is this different from the other experiments?} %The type of context was randomly sampled, resulting in six main trials with basic-level contexts and six main trials with subordinate contexts per participant.
\begin{figure}[t]
	\begin{center}
		\includegraphics[width=\linewidth]{expt3-cc-inference-revised.pdf}
	\end{center}
	\vspace{-0.5cm}
	\caption{Experiment 3 results. Means and 95\% bootstrapped confidence intervals of inferred basic-level comparison class proportions (e.g.,~“...big relative to other dogs”).Context strongly modulated the comparison class (left~vs.~right panel). The noun additionally provided a cue to the comparison class (red~vs.~blue) bars, even in subject position. There is a hint of an interaction of the noun (red~vs.~blue) with the syntax. 
	}
	\vspace{-0.3cm}
	\label{cc-inference}
\end{figure}

\subsubsection{Results} 
Responses were categorized as either basic-level and subordinate target labels. 
Six of participants' responses were superordinate category labels (e.g., ``animals''), which we collapsed with the basic-level responses. 
39 uncategorizable responses (1.6\%) were excluded from the analysis. 

To test our pre-registered predictions, we constructed a Bayesian logistic mixed-effects regression model that predicted the response category (basic-~vs.~subordinate-level labels) from the syntax, context, the noun and the pair-wise two-way and three-way interactions, with a maximal random effects structure by-participant and by-referent \cite{barr2013}.\footnote{ \texttt{response\_category $\sim$ syntax*NP*context + (1 + syntax*NP*context || subject) + (1 + syntax*NP*context || target)}. We set the correlation of random effects to be 0, for computational tractability.}
A simple, syntactic account of comparison class determination would hold that the noun in the sentence determines the comparison class, and so the same sentence should receive the same comparison class regardless of the context.
Contra this account, we observe a large main effect of context: more basic-level comparison classes were inferred from the basic-level than the subordinate-level context, even given the exact same sentence ($\beta = \rlgetnum{expt3_full_brm_revised.csv}{Rowname}{context_sum}{Estimate}{2} [\rlgetnum{expt3_full_brm_revised.csv}{Rowname}{context_sum}{l.95..CI}{2},  \rlgetnum{expt3_full_brm_revised.csv}{Rowname}{context_sum}{u.95..CI}{2}]$; Fig.~\ref{cc-inference}, left~vs.~right facets).
We additionally observe a main effect of noun-label, regardless of the syntactic position of the noun, arguing against an account of comparison class determination wherein only syntacically-modified nouns (i.e., ADJ NOUN) provide the comparison class: basic-level labels receiving more basic-level comparison classes than the anaphoric \emph{one} ($\beta = 1.23 [0.3, 2.2]$) and subordinate-level labels receiving fewer basic-level comparison classes than \emph{one} ($\beta = -1.57 [-0.6, -2.6]$). \mht{replace with numbers read in from .csv... also, add basic vs. sub contrast}.
Furthermore, we see that a subordinate noun can be the minority response even when it is syntactically modified by the adjective (e.g., ``big Great Dane'' $\rightarrow$ big for a dog; Fig.~\ref{cc-inference}, left-facet, blue-bar). \pt{should we mention 'one'specifically once more?}



%In addition to a global preference for basic-level comparison classes ($\beta = \rlgetnum{expt3_brm.csv}{Rowname}{Intercept}{Estimate}{2} [\rlgetnum{expt3_brm.csv}{Rowname}{Intercept}{l.95..CI}{2},  \rlgetnum{expt3_brm.csv}{Rowname}{Intercept}{u.95..CI}{2}]$), 

%We see that the basic noun vs. subordinate noun difference was maintained even in the subject position (Fig.~\ref{cc-inference}; red~vs.~blue bars), consistent with the inferential account in which the choice of noun is a cue to a speaker's conceptualization of the referent.

We see some moderate evidence in support of the Noun x Syntax interaction of the kind found in Expts.~1 \& 2 (Fig.~\ref{cc-inference}; red~vs.~blue bars vs. x-axis; 94.9\% of the posterior distribution of the interaction was greater than 0, analogous to a one-tailed test). 
To further explore the Noun x Syntax interaction, we built a regression model that assumed only a fixed-effect of context \footnote{\texttt{response\_category $\sim$ syntax*NP + context + (1 + syntax*NP || subject) + (1 + syntax*NP || item) }} and confirmed the NP (basic~vs.~sub) x Syntax interaction ($\beta = -0.49 [-0.86, -0.05]$).
Examining the syntax interaction in the context of NP~vs.~\emph{one} contrast, we found that 95.1\% of the posterior distribution of subordinate-NP~vs.~\emph{one} x Syntax interaction was less than 0 ($\beta = -0.38 [-0.84, 0.07]$) whereas only 62\% of the posterior of the basic-NP~vs.~\emph{one} X Syntax interaction was greater than 0 ($\beta = 0.62 [-0.42, 0.57]$), a suggestion of a difference we return to in the discussion. 

%in basic-level context, basic-level comparison classes were more likely to be inferred from the basic-level NPs than from 'one' ($\beta = \rlgetnum{expt3_brm.csv}{Rowname}{NP_basic}{Estimate}{2} [\rlgetnum{expt3_brm.csv}{Rowname}{NP_basic}{l.95..CI}{2}, \rlgetnum{expt3_brm.csv}{Rowname}{NP_basic}{u.95..CI}{2}]$) and subordinate comparison classes from subordinate NPs compared to 'one' ($\beta = \rlgetnum{expt3_brm.csv}{Rowname}{NP_sub}{Estimate}{2} [\rlgetnum{expt3_brm.csv}{Rowname}{NP_sub}{l.95..CI}{2}, \rlgetnum{expt3_brm.csv}{Rowname}{NP_sub}{u.95..CI}{2}]$). 
%However, the NP x Syntax interactions were not appreciably different from 0 and neither were the three way Context x NP x Syntax interactions. 
%As a manipulation check, we find the inferences drawn from the NP ``one'' were strongly affected by the visual context: in the basic-level context, only the basic-level comparison class was inferred. In contrast, although the subordinate context pointed to the subordinate comparison class, both subordinate-level and basic-level comparison classes were inferred, consistent with a bias towards basic-level comparison classes.
%In an exploratory analysis that assumes only a fixed-effect of context , we find a significant interaction of syntax and subordinate NP ($\beta = \rlgetnum{expt3_2_way_brm.csv}{Rowname}{syntax_contr:NP_sub}{Estimate}{2} %[\rlgetnum{expt3_2_way_brm.csv}{Rowname}{syntax_contr:NP_sub}{l.95..CI}{2}, \rlgetnum{expt3_2_way_brm.csv}{Rowname}{syntax_contr:NP_sub}{u.95..CI}{2}]$); that is, across contexts, the subordinate NP appearing in the predicate position has a stronger influence on the comparison class than when it appears in the subject position (Fig~\ref{cc-inference} green~vs~blue bars). 
%The same interaction was not observed for the basic-NP ($\beta = \rlgetnum{expt3_2_way_brm.csv}{Rowname}{syntax_contr:NP_basic}{Estimate}{2} [\rlgetnum{expt3_2_way_brm.csv}{Rowname}{syntax_contr:NP_basic}{l.95..CI}{2}, \rlgetnum{expt3_2_way_brm.csv}{Rowname}{syntax_contr:NP_basic}{u.95..CI}{2}]$; Fig~\ref{cc-inference} red~vs~green bars).
%\mht{more analysis restricting to basic-NP vs. sub-NP. and subsetting by context}
%In the basic-level visual context, the subordinate NP provided a slightly stronger cue to the comparison class (i.e., more subordinate comparison classes) when the NP appeared in the predicate than when it appeared in the subject of the sentence; a parallel asymmetry could not be observed for the basic-level label, because basic-level comparison class inferences were already at ceiling as measured by the ``one''-NP condition. \mht{what stats to show? }
% in the 3-way analysis, there is a significant effect of basic-NP ... in the 2-way analysis, there is none
%We see a similar pattern of results in the subordinate-level visual context, wherein the subordinate NP provided a stronger cue to the comparison class when it appeared in predicate than in the subject position. 
%This interaction is confirmed by a preliminary regression analysis that assumes no context interactions. 
%Interestingly, we do not observe the parallel effect with the basic-level NP (i.e., more basic-level comparison classes) in the subordinate-level context, even though the probability of a basic-level comparison class is not at ceiling (again, as indexed by the ``one''-NP condition).
%Further supporting our hypotheses, in the subordinate context more participants inferred the subordinate comparison class from the subordinate NP in the predicate compared to the subject position since the proportion of subordinate comparison classes inferred from the context was not maximal. 
%The basic-level NP provides a more general comparison class than the subordinate context, such that a %higher proportion of inferred comparison classes was basic-level given the basic-level NP in predicate position than given the NP 'one'.

\section{Discussion}
Understanding language requires appreciating the context in which the words are uttered.
Yet, speakers almost never explicitly articulate what features of context are relevant, but leave it to listeners to pragmatically reconstruct. 
Inferring comparison classes for relative adjectives (e.g., \emph{big}) is a case study in this larger phenomenon of pragmatic reconstruction of context.
Comparison classes are employed for understanding gradable adjectives \cite<e.g., \emph{big};>{Kennedy2007} and many other linguistic expressions that convey relative meanings, including vague quantifiers \cite<e.g., ``She ate \emph{a lot} of hot dogs'';>{scholler2017semantic} and generic language \cite<e.g., ``Dogs are friendly \emph{[relative to other animals]}'';>{tessler2019language}.

%In this paper, we find that listeners use syntactic cues and world knowledge to adjust the comparison class. 
The basic inference we measure is that listeners are more likely to use the noun phrase in the sentence as the comparison class when the noun appears in the predicate (``That's a big Great Dane'') than in the subject of the sentence (``That Great Dane is big''). We propose an information-structural reason for this inference: When the noun is in the subject of the sentence (especially when it combines with the deictic ``That''), its usage can be explained away by its utility in reference, whereas a predicate-noun less strongly conveys reference and hence is more likely to be produced by a speaker aiming to convey the comparison class. 
%Across three diverse dependent measures (Syntax Rating, Noun Production, Comparison Class Inferences), we found converging evidence for such an effect. 
%More broadly, these results imply that semantic rules might in part be determined by constructions which pair linguistic form and informational function.  

The reference-predication trade-off hypothesis provides a starting point for an integrative account for understanding how diverse contextual circumstances and cues drive inferences about the relevant aspects of context for understanding a message (in our case, the comparison class). 
We argue that the utility of a noun phrase for reference can be modulated based on the syntactic position of the noun, but the syntactic distinction of subject~vs.~predicate is just one cue for referential~vs.~predicative uses. %Predicate nouns can be referential.
In Expt.~3, we found that comparison class inferences were driven by a subordinate noun (in comparison to the noun ``one'') more so when the noun appeared in the predicate of the sentence than when it appeared in the subject. \pt{can we call 'one' a different name than noun?}
We hypothesized this effect is because the referential utility of the subordinate-noun differs by syntactic position, but we note that the context must also support this inference.
The referential utility of the basic-level noun was not affected by syntactic position because in neither context (basic or subordinate) was the basic-noun an informative referring expression: \emph{dog} is both uninformative in a context of \emph{dogs} (basic-level context) and the context of \emph{Great Danes} (subordinate-level context).
Because the basic-noun is uninformative as a referring expression, the referential-predicative trade-off view would not expect comparison class inferences to differ across syntactic positions for the basic-NP label, which is indeed what we found. \pt{do we want to mention anything about the higher basic-level inference rate for basic Ns in subordinate context?}
Further tests of this account should experimentally manipulate the referential utility of the NP \cite<e.g., using \emph{dog} in the context of other animals;>{graf2016animal} and confirm its impact on inferences about the comparison class. 

Our subject~vs.~predicate noun position manipulation is perfectly confounded with whether the adjective directly syntactically modifies the noun~vs.~not. Direct modification could occur in the subject of the sentence: ``That big Great Dane is mine''. The reference-predication hypothesis we described here would predict that even in this sentence structure, the fact that Great Dane is likely to be used referentially takes some weight off its utility as a comparison class setting noun. We plan to explore this prediction in a follow-up experiment. 

The reference-predication distinction we highlight in this paper, and look at through the lens of grammatical subject~vs.~predicate, is similar to the distinction of discourse-given~vs.~discourse-novel \pt{noun missing?} or topic~vs.~comment from Information Structure. Though the precise definitions of topic~vs.~comment are debated \cite<e.g.,>{jacobs2001dimensions}, the broad distinction is that \emph{topic} is what is being talked about or what is given and \emph{comment} is what is being said of the topic or what is new \cite{lambrecht1996information, krifka2008basic}. We believe this distinction is dissociable from that of reference~vs.~predication that we focus on in this paper \cite{Reboul2001}. Consider, for example, the sentence: ``What's big is that Great Dane''. The sentence seems appropriate in a context where the topic---what is given---is that something is big, and the comment---what's new---is ``that Great Dane''. Yet, ``Great Dane'' also seems to be establishing reference, and additionally striking, it is doing so from the predicate of the sentence. Thus, though we examine reference~vs.~predication through the subject-predicate distinction, we believe the communicative goals are primary in driving inferences about the comparison class and are distinct from the topic-comment distinction from Information Structure.% repetition of the third sentence before in the paragrpah before 

Understanding comparison classes is a basic cognitive skill for interpreting a simple class of context-sensitive expressions: scalar adjectives.
Very soon after children start producing their first scalar adjective---e.g., \emph{big}---they seem to understand its context sensitive behavior and can flexibly switch between contexts
\cite<e.g., that a mitten, which is a small mitten, might also be \emph{big for a doll};>{ebeling1994children}.
The kinds of cues that have been shown to modulate comparison class inferences in young children have been rather dramatic cues (e.g., ``is the mitten big for the doll?''), though 2-year-olds appear sensitive to the specificity of the noun alone when interpreting adjectives \cite{Mintz2002}.
The problem that the language learner faces goes beyond inferring the comparison class in the moment: Young children are jointly learning the meaning of the nouns and adjectives along with trying to construct the appropriate comparison classes to interpret the utterances they hear.
Understanding children's sensitivity to the cues we investigate here can provide some hints as to how they are able to accomplish the incredible feat of learning language.
% \section{Acknowledgments}
\bibliographystyle{apacite}

\setlength{\bibleftmargin}{.125in}
\setlength{\bibindent}{-\bibleftmargin}

\bibliography{refpred-cogsci2020}


\end{document}
